\begin{abstract}
We apply equational reasoning, a core principle of functional programming, to design a Str8ts-solving algorithm. Despite its similarities to Sudoku, research on this puzzle is scarce. A Google Scholar search for “str8ts solver” returns only eight results, all of which are false positives. We follow the approach outlined in Bird’s \textit{A Simple Sudoku Solver}. First, we develop a basic solver, \textit{simpleSolve}. Then, we systematically refine it into a more efficient one, \textit{advancedSolve}. By leveraging structural similarities between Str8ts and Sudoku, we can reuse most of Bird’s transformation steps directly. We present our work as suggested in \textit{How To Write A Functional Pearl}.

To benchmark performance, we set up a Haskell project using GHCup for tool management, Stack as the build tool, and Visual Studio Code as the editor. We introduced practical modules, PuzzleDownloader and RuntimeAnalyser. The source code is available in our GitHub repository at \href{https://github.com/e11720061/a-simple-str8ts-solver/tree/main/project}{/project}, and can be used to test other solutions. We selected 365 puzzles from the official website, \href{https://www.str8ts.com/}{www.str8ts.com}, published between 01.01.2023 and 31.12.2023. Running \textit{advancedSolve} on these puzzles on a 16-inch MacBook Pro 2023 with an M3 Max chip establishes a reproducible baseline for future research. Our algorithm solved all puzzles with an average runtime of 504.1 ms, a minimum of 3.5 ms, and a maximum of 14,359.3 ms ($\approx$ 14.4 s). (Results rounded to one decimal place.)
\end{abstract}