\chapter{Future Work}\label{chap:future-work}

We propose several directions for future work. First, more advanced pruning strategies could be explored. In this thesis, we presented only a basic approach. However, the official Str8ts website lists fifteen strategies in total~\cite{website-str8ts}. Unlike our approach, some of these strategies do not rely on fixed choices, while others analyze more complex patterns within the grid. For example, \textit{Setti’s rule} states that each digit appears in the same number of rows and column. Future research could investigate which combination of strategies yields the best performance.

Another promising direction is the extension of run-time analysis. The function $advancedSolve$ could be compared with the five Haskell solvers identified during our GitHub search. Since all of these solvers were last updated in 2021 or earlier, their functionality and correctness should first be verified. Once validated, their performance could be measured using the framework provided in this thesis. Because $readAndAnalyzePuzzles$ expects a solver of type $\textbf{Solver} = Grid \to [Grid]$, some mapping may be necessary to accommodate different solver interfaces.

Additionally, $readAndAnalyzePuzzles$ could be modified to accept multiple solvers simultaneously, i.e., $solvers :: [Solver]$, instead of a single $solver :: Solver$. This modification would eliminate the need to restart the program for each solver, and results could be consolidated into a single Excel file rather than separate files that must be merged manually, as in the current implementation.

Finally, a code analysis module could be developed. Selecting appropriate metrics—whether qualitative or quantitative—poses a challenge, particularly when attempting to assess the influence of different programming paradigms while minimizing bias from individual coding styles. Research on structured code comparison could provide guidance for this process. Similar to the run-time analysis, this procedure could be automated, potentially through a new module named \textit{CodeAnalyser.hs}.