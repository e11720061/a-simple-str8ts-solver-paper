\chapter{Future Work}\label{chap:future-work}

We propose several directions for future work. First, researchers can explore more advanced pruning strategies. In this thesis, we presented only a basic approach. However, the official Str8ts website lists nine strategies in total~\cite{website-str8ts}. Unlike our approach, some of these strategies avoid fixed choices or analyze more complex patterns within the grid. \textit{Setti’s rule}, for example, examines rows and columns simultaneously. Future research can determine which combination of strategies yields the best performance.

Another promising direction is to extend our run-time analysis. We can compare the $advancedSolve$ function with the five Haskell solvers we identified in our GitHub search. Because all of these solvers last received updates in 2021 or earlier, researchers should first verify their functionality and correctness. After validation, they can measure performance using the framework we provided in this thesis. Since the framework expects algorithms of $\textbf{type}~Solver = Grid \to [Grid]$, researchers may need to perform some mapping.

Additionally, we can modify our framework to accept multiple solvers simultaneously, i.e., $solvers::[Solver]$, instead of $solver::Solver$. This modification eliminates the need to restart the program for each solver. It allows the framework to consolidate results into one Excel file rather than separate files that require manual merging.

Finally, researchers can conduct a code analysis. As with run-time analysis, we can automate this procedure through a new module named CodeAnalyser. However, selecting appropriate metrics -- whether qualitative or quantitative -- remains a challenge. Especially when assessing the influence of different programming paradigms while minimizing bias from individual coding styles. Research on structured code comparison can guide this process.